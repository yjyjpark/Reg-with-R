% Options for packages loaded elsewhere
\PassOptionsToPackage{unicode}{hyperref}
\PassOptionsToPackage{hyphens}{url}
%
\documentclass[
]{book}
\usepackage{amsmath,amssymb}
\usepackage{lmodern}
\usepackage{iftex}
\ifPDFTeX
  \usepackage[T1]{fontenc}
  \usepackage[utf8]{inputenc}
  \usepackage{textcomp} % provide euro and other symbols
\else % if luatex or xetex
  \usepackage{unicode-math}
  \defaultfontfeatures{Scale=MatchLowercase}
  \defaultfontfeatures[\rmfamily]{Ligatures=TeX,Scale=1}
\fi
% Use upquote if available, for straight quotes in verbatim environments
\IfFileExists{upquote.sty}{\usepackage{upquote}}{}
\IfFileExists{microtype.sty}{% use microtype if available
  \usepackage[]{microtype}
  \UseMicrotypeSet[protrusion]{basicmath} % disable protrusion for tt fonts
}{}
\makeatletter
\@ifundefined{KOMAClassName}{% if non-KOMA class
  \IfFileExists{parskip.sty}{%
    \usepackage{parskip}
  }{% else
    \setlength{\parindent}{0pt}
    \setlength{\parskip}{6pt plus 2pt minus 1pt}}
}{% if KOMA class
  \KOMAoptions{parskip=half}}
\makeatother
\usepackage{xcolor}
\usepackage{longtable,booktabs,array}
\usepackage{calc} % for calculating minipage widths
% Correct order of tables after \paragraph or \subparagraph
\usepackage{etoolbox}
\makeatletter
\patchcmd\longtable{\par}{\if@noskipsec\mbox{}\fi\par}{}{}
\makeatother
% Allow footnotes in longtable head/foot
\IfFileExists{footnotehyper.sty}{\usepackage{footnotehyper}}{\usepackage{footnote}}
\makesavenoteenv{longtable}
\usepackage{graphicx}
\makeatletter
\def\maxwidth{\ifdim\Gin@nat@width>\linewidth\linewidth\else\Gin@nat@width\fi}
\def\maxheight{\ifdim\Gin@nat@height>\textheight\textheight\else\Gin@nat@height\fi}
\makeatother
% Scale images if necessary, so that they will not overflow the page
% margins by default, and it is still possible to overwrite the defaults
% using explicit options in \includegraphics[width, height, ...]{}
\setkeys{Gin}{width=\maxwidth,height=\maxheight,keepaspectratio}
% Set default figure placement to htbp
\makeatletter
\def\fps@figure{htbp}
\makeatother
\setlength{\emergencystretch}{3em} % prevent overfull lines
\providecommand{\tightlist}{%
  \setlength{\itemsep}{0pt}\setlength{\parskip}{0pt}}
\setcounter{secnumdepth}{5}
\usepackage{booktabs}
\ifLuaTeX
  \usepackage{selnolig}  % disable illegal ligatures
\fi
\usepackage[]{natbib}
\bibliographystyle{plainnat}
\IfFileExists{bookmark.sty}{\usepackage{bookmark}}{\usepackage{hyperref}}
\IfFileExists{xurl.sty}{\usepackage{xurl}}{} % add URL line breaks if available
\urlstyle{same} % disable monospaced font for URLs
\hypersetup{
  pdftitle={회귀분석},
  pdfauthor={박동련},
  hidelinks,
  pdfcreator={LaTeX via pandoc}}

\title{회귀분석}
\author{박동련}
\date{2023-02-23}

\begin{document}
\maketitle

{
\setcounter{tocdepth}{1}
\tableofcontents
}
\hypertarget{uxc18cuxac1cuxd558uxae30}{%
\chapter*{소개하기}\label{uxc18cuxac1cuxd558uxae30}}
\addcontentsline{toc}{chapter}{소개하기}

\hypertarget{ch1}{%
\chapter{회귀와 모형 설정}\label{ch1}}

사회현상, 자연현상, 경영현상이나 경제현상 등을 측정한 변수들의 변동은 그 현상과 관련된 여러 변수들의 영향에 의해서 발생될 수 있다고 볼 수 있다.
해당 분야 분석자는 어떤 현상을 과학적 시각으로 셜명하기 위해서 관련된 변수들 사이의 관계를 분석하여 변동의 상당 부분을 설명하기 위한 적절한 모형을 설정하고자 할 것이다.
회귀분석(regression analysis)이란 특정 현상과 그 현상에 영향을 미칠 수 있는 변수들 사이의 관계를 분석하고 모형화하기 위한 통계적 기법이다.
회귀분석에서는 특정 현상과 그 현상에 영향을 미칠 수 있는 변수들 사이의 함수관계를 표현할 수 있는 모형을 이론적 근거나 경험적 판단에 의해서 설정하고, 관측된 자료에 의해서 함수관계를 추정해서 변수들 사이의 관계를 설명하거나 예측하는데 사용된다.
따라서 회귀분석은 사회과학, 공학, 물리학, 생물학, 경영학, 경제학, 의학 등 거의 모든 학문 분야에서 가장 널리 사용되고 있는 통계적 분석 방법인 것이다.

회귀모형을 설정할 때 우리가 관심이 있는 특정 현상의 변동을 나타내는 변수를 종속변수(dependent variable) 혹은 반응변수(response variable)라고 하고, \(Y\)로 표시한다.
반응변수의 변동에 영향을 줄 것으로 보이는 변수를 독립변수(independent variable) 혹은 설명변수(explanatory variable)라고 하고, \(X\)로 표시한다.
만일 설명변수의 수가 \(k\)개 있다고 하면 \(X_{1}, X_{2}, \ldots, X_{k}\)로 표시한다.

예를 들어, 회사 마케팅 부서에서 특정 제품의 매출액 변동을 예측할 수 있는 모형을 설정하고자 할 때 제품의 매출액 변동에 영향을 미칠 수 있는 변수들로 광고비 지출액, 가격수준, 기술수준, 디자인 선호수준, 소득수준, 영엽사원의 수, 경쟁제품의 수 등을 사용하게 되는데, 이 경우에 반응변수는 제품의 매출액이 되고, 광고비 지출액 등은 설명변수가 된다.

또 다른 예로써 투자자문회사에서 고객의 투자행위에 대한 예측모형을 개발할 때 투자행위에 영향을 미칠 수 있는 변수들로 연소득 수준, 미래 경제상황지수, 시장 이자율 등을 이용하게 되는데, 여기에서 반응변수는 고개들의 투자액이 되고, 연소득 수준 등은 설명변수가 된다.

회귀모형은 모형에 포힘되는 설명변수가 한 개인 경우에는 단순회귀모형(simple regression model)이라고 하고, 두 개 이상의 설명변수가 모형에 포함되는 경우에는 다중회귀모형(multiple regression model)이라고 한다.

본 장에서는 회귀모형의 기본 개념으로 다음의 내용을 설명하도록 하겠다.

\begin{itemize}
\item
  반응변수와 설명변수 사이의 관계 설정
\item
  두 변수 사이의 관계를 대략적으로 파악할 수 있는 산점도(scatter plot) 작성과 활용
\end{itemize}

\hypertarget{uxbcc0uxc218uxb4e4-uxc0acuxc774uxc758-uxad00uxacc4}{%
\section{변수들 사이의 관계}\label{uxbcc0uxc218uxb4e4-uxc0acuxc774uxc758-uxad00uxacc4}}

  \bibliography{book.bib}

\end{document}
